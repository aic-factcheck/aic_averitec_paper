%!TEX ROOT=../emnlp2023.tex
\section{Other Examined Approaches}
In this section, we also describe a third, optional module we call the \textit{veracity classifier}, which takes the claim and its evidence generated by our evidence \& label generator~(section~\ref{sec:generation}) and predicts the veracity label independently, based on the suggested evidence, using a fine-tuned NLI model.
We also describe the options of its ensembling with veracity labels predicted in the generative step (section~\ref{likert}).

The absence of a dedicated veracity classifier has not been shown to decrease the performance of our pipeline significantly (as shown, e.g., in tables~\ref{tab:pipeline_scores} and~\ref{tab:nli}) so we suggest to omit this step altogether and we proceed to participate in the \averitec{}  shared task without it, proposing a clean and simple RAG pipeline without the extra step (Figure~\ref{fig:pipeline}) for the fact-checking task.

\subsection{Single-evidence Classification with Label Aggregation}
In the earliest stages of experimenting, we utilized the baseline classifier provided by \averitec{} authors\footnote{https://huggingface.co/chenxwh/AVeriTeC}~\cite{averitec2024}.
It is based on the BERT~\cite{devlin-etal-2019-bert} and was further fine-tuned on the \averitec{}  dataset~\cite{averitec2024}. 
It takes one claim and one question-answer evidence as input -- each claim therefore has multiple classifications, one for each evidence. The classifications are then aggregated using a heristic of several if-clauses to determine the final label. 

We experiment with altering this heuristic (e.g. by making \textit{not enough evidence} the final label only when no other labels are present at any evidence), and training NLI models that could work better with it, such as 3-way DeBERTaV3~\cite{he2023debertav3improvingdebertausing} without a breakthrough result, motivating a radically different approach.

\subsection{Multi-evidence Classification}
\label{subsubsec:concatenation}
The multi-evidence approach is to fine-tune a 4-way Natural Language Inference (NLI) classifier, using the full scope of evidence directly at once, without heuristics.
For that, we concatenate all of the evidence together using a separator \texttt{[SEP]} token. This allows the model to know exact question-answer borders, albeit using a space has turned out to be just as accurate as the experiments went on. As the veracity verdict should be independent of the evidence ordering, we also experiment with sampling different permutations in the fine-tuning step to increase the size of our data.

We carry out the fine-tuning using the \averitec{} train split with gold evidence and labels on DeBERTaV3~\cite{he2023debertav3improvingdebertausing} in two variants: the original large one\footnote{https://huggingface.co/microsoft/deberta-v3-large} and one pre-finetuned on NLI tasks\footnote{https://huggingface.co/cross-encoder/nli-deberta-v3-large}, and also Mistral-7B-v0.3 model\footnote{https://huggingface.co/mistralai/Mistral-7B-v0.3} with a classification head (MistralForSequenceClassification) provided by the Huggingface Transformers library~\cite{wolf-etal-2020-transformers} that utilizes the last token. In the preliminary testing phase, the original DeBERTaV3 Large performed the best and was used in all other experimental settings.

From the approaches described above, we achieved the best results without permuting the evidence, achieving 0.71 macro $F_1$ score using a space-separation. The \texttt{[SEP]} model achieved a comparable 0.70 macro $F_1$ score, and the random order model performed worse with a 0.67 macro $F_1$ score, all improving significantly upon baseline, yet falling behind the capabilities of generating the labels alongside evidence in a single chain-of-thought. 
We provide our best DeBERTaV3 finetuned model publicly in a Huggingface repository\footnote{https://huggingface.co/ctu-aic/deberta-v3-large-AVeriTeC-nli}.

\subsection{Ensembling Classifiers}
\label{subsubsec:ensembling}

Encouraged by the promising results of our multi-evidence classifiers, we go on to try to ensemble the models with LLM predictions from section~\ref{likert}, using a weighted average of the class probabilities of our models.
We have experimented with multiple weight settings: 0.5:0.5 for even votes, 0.3:0.7 in favour of the LLM to exploit its accuracy while tipping its scales in cases of a more spread-out label probability distribution, as well as 0.1:0.9 to use the fine-tuned classifier only for tie-breaking, listing the results in Table~\ref{tab:nli}.

We also tried tuning our ensemble weights based on a subset of the dev split, without a breakthrough in accuracy on the rest of dev samples.

The last method we tried was stacking using logistic regression. However, this setup classified no labels from \textit{Not Enough Evidence} and \textit{Conflicting Evidence/Cherrypicking}, and we could not achieve reasonable results. For logistic regression, we used the scikit-learn library~\cite{scikit-learn}.

We conclude that the augmentation of the pipeline from figure~\ref{fig:pipeline} with a classification module using a single NLI model or an ensemble with LLM is unneccessary, as it adds complexity and computational cost without paying off on the full pipeline performance (table~\ref{tab:pipeline_scores}).

\subsection{Conflicting Evidence/Cherrypicking Detection}

During the experiments, we discovered that classifying the \textit{Conflicting Evidence/Cherrypicking} class is the most challenging task, achieving a near-zero $F_1$-score across our various prototype pipelines.
To overcome this problem, we tried to build a binary classifier that would output one if the input claim is cherrypicked and 0 otherwise. We tried to use the DeBERTaV3 Large model with basic cross-entropy loss (other experimental settings were the same as in section~\ref{subsubsec:concatenation}), but it could not pick up the training task due to the \textit{Conflicting Evidence/Cherrypicking} underrepresentation in train set -- less than 7\% of the samples carry the label. 

Even after exploring various other methods, we did not get a reliable detection scheme for this task, perhaps motivating a future collection of data that represents the class better.
While writing this system description paper, we found an interesting research by~\citet{jaradat2024contextawaredetectioncherrypickingnews} that uses a radically different approach to detect cherrypicking in newspaper articles.

    \centering
    \setlength\tabcolsep{3pt} % default value: 6pt
    \resizebox{\columnwidth}{!}{%
    \begin{tabular}{lcccc}
        \toprule
        \textbf{Classifier} & \textbf{Acc} & \textbf{$F_1$} & \textbf{Prec.} & \textbf{{\footnotesize {Recall}}} \\
        \midrule
        GPT4o & \textbf{0.72} & 0.46 & 0.48 & 0.47 \\
        Claude 3.5 Sonnet & 0.64 & 0.49 & 0.50 & 0.52 \\
        DeBERTa & 0.63 & 0.39 & 0.40 & 0.41 \\
        DeBERTa - random@10 & 0.65 & 0.41 & 0.41 & 0.44 \\
        $0.5\cdot\mbox{DeBERTa}+0.5\cdot\mbox{GPT4o}$ & 0.70 & 0.43 & 0.41 & 0.45 \\
        $0.5\cdot\mbox{DeBERTa}+0.5\cdot\mbox{Claude}$ & 0.68 & 0.47 & 0.50 & 0.49 \\
        $0.3\cdot\mbox{DeBERTa}+0.7\cdot\mbox{GPT4o}$ & \textbf{0.72} & 0.45 & 0.45 & 0.46 \\
        $0.3\cdot\mbox{DeBERTa}+0.7\cdot\mbox{Claude}$ & 0.66 & \textbf{0.50} & \textbf{0.51} & \textbf{0.53} \\
        $0.1\cdot\mbox{DeBERTa}+0.9\cdot\mbox{GPT4o}$ & \textbf{0.72} & 0.39 & 0.46 & 0.43 \\
        $0.1\cdot\mbox{DeBERTa}+0.9\cdot\mbox{Claude}$ & 0.64 & 0.49 & 0.50 & 0.54 \\
        \midrule
        Llama 3.1 & 0.73 & 0.44 & 0.43 & 0.46 \\
        \bottomrule
    \end{tabular}
    }
    \caption{Evalution of the classifiers on the development set. $F_1$, Precision and Recall are computed as macro-averages. The random@10 suffix indicates that the classifier was tuned using 10 different random orders of QA pairs for each claim. GPT4o stands for the Likert classifier based on GPT-4o, Claude 3.5 Sonnet is the Likert classifier based on Claude 3.5 Sonnet, and DeBERTa is classifier based on DeBERTaV3 Large fine-tuned on \averitec{} gold evidence and labels.}
    \label{tab:nli}
