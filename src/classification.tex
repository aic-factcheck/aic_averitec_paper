%!TEX ROOT=../emnlp2023.tex

\subsection{Classification Step}
In this section, we describe the classification component of our pipeline. First, we introduce chronologically the different approaches, the ensembles we tried, and their pros and cons. Then, we present evaluation results across different methods and various metrics. The goal of this step is to take the input claim and questions with answers provided by previous stages \todo{ref} and classify the claim into one of the four classes: \textit{Supported}, \textit{Refuted}, \textit{Not Enough Evidence}, or \textit{Conflicting Evidence/Cherrypicking} as defined by~\citealp{averitec2024}.

\subsubsection{Approaches}
\todo{Describe what was done +  why}

In the earliest stages of experimenting, we utilized the classifier from baseline provided by authors\footnote{https://huggingface.co/chenxwh/AVeriTeC}~\cite{averitec2024}. This classifier is based on the BERT~\cite{devlin-etal-2019-bert} model and was further fine-tuned on the AVeriTeC dataset~\cite{averitec2024}. It takes one claim and one question with its answer as input. The output of this encoder model (class logits) is then further processed by several if-clauses to determine the final label. To the best of our knowledge, the complete classifier outputs \textit{Not Enough Evidence} whenever the model predicts the \textit{Not Enough Evidence} or some 4th label (here, we are not sure how the model was precisely fine-tuned) for any of the (up to 10) question-answer pairs. It predicts \textit{Supported} if there is no \textit{Not Enough Evidence} prediction and at least one \textit{Supported} prediction and no \textit{Refuted} prediction. Then it predicts \textit{Refuted} if there are no \textit{Not Enough Evidence} or \textit{Supported} predictions and at least one \textit{Refuted} prediction. Otherwise, it predicts \textit{Conflicting Evidence/Cherrypicking}. Despite considerable effort, we could not understand why and how the model was trained on four labels and exact background of this, at least from our point of view, ad-hoc post-processing logic.

We argue that the post-processing logic should be changed so that \textit{Not Enough Evidence} is predicted only if there is no other label to predict. We think that, for example, if among the predictions for question-answer pairs, there is one \textit{Not Enough Evidence} prediction and the rest are \textit{Supported} predictions, the final prediction should be \textit{Supported} and not \textit{Not Enough Evidence} as in the original logic. We did implement this change (see listing~\ref{lst:post-processing}), and surprisingly, it did not improve the results as the number of \textit{Not Enough Evidence} class has fallen significantly \todo{numbers}. Despite that, we still think this change is more logical and should be used in the future.

\lstset{
    language=Python,
    basicstyle=\ttfamily\footnotesize\linespread{0.9}, % Smaller font with less spacing
    keywordstyle=\color{blue}\bfseries,
    commentstyle=\color{green!50!black}\itshape,
    stringstyle=\color{orange},
    numberstyle=\tiny\color{gray},
    numbers=left, % Line numbers on the left
    stepnumber=1, % Line numbers for every line
    numbersep=5pt, % Space between line numbers and code
    tabsize=4, % Size of tabs
    showstringspaces=false, % Don't show spaces in strings
    breaklines=true, % Line wrapping
    breakatwhitespace=true,
    frame=lines, % Add a frame around the code
    captionpos=b, % Caption at the bottom
}
\begin{figure}
    \begin{lstlisting}[language=Python, frame=single]
if has_true and has_false:
  answer = 3
elif has_true and not has_false:
  answer = 0
elif not has_true and has_false:
  answer = 1
else:
 answer = 2 #otherwise NEI
        \end{lstlisting}
    \caption{Our proposed post-processing logic} 
    \label{lst:post-processing}       
\end{figure}

Moreover, we think the model should be trained only on three labels because only three are used. We tried our fine-tuning of a newer encoder model DeBERTaV3~\cite{he2023debertav3improvingdebertausing} on only three labels, and the results were better \todo{numbers}.

The results of our preliminary experiment were not satisfactory, and we decided to try more radically different approaches, as described in the following sections.

\subsubsubsection*{Concatenation and Four Classes Classifier}
\label{subsubsec:concatenation}
\todo{better name}
\todo{describe, models, Concatenation, sep-concatenation, random order, four classes!, Mistral, DeBERTaV3, metion Transformers-HF}
Our first approach is to finetune a text classification model on all four classes directly. However, this means that as evidence, we must feed the model with all question-answer pairs at once to classify \textit{Conflicting Evidence/Cherrypicking} correctly. For that, we tried several options. First, we concatenated all of them using single blank space in the order they are provided in the dataset. This should work because we expect that annotators created the question-answer pairs usually in a specific order. The second option we tried is concatenation using a separator \texttt{[SEP]} token again in the original order. This option utilizes the model's knowledge about separator tokens and allows it to know exact question-answer borders. The last option we tried is to randomly select more orders of the question-answer pairs (all combinations computation is not feasible - up to $10!$ of combinations).

As models, we chose DeBERTaV3~\cite{he2023debertav3improvingdebertausing} in two variants: the original large one\footnote{https://huggingface.co/microsoft/deberta-v3-large} and one pre-finetuned on NLI tasks\footnote{https://huggingface.co/cross-encoder/nli-deberta-v3-large}, and also Mistral-7B-v0.3 model\footnote{https://huggingface.co/mistralai/Mistral-7B-v0.3} with a classification head (MistralForSequenceClassification) provided by the Huggingface Transformers library~\cite{wolf-etal-2020-transformers} that utilizes the last token. In the preliminary testing phase, the original DeBERTaV3 Large performed the best and was used in all other experimental settings.

From the approaches described above, we achieved the best results with the original order achieving 0.71 macro $F_1$ score. The separator model achieved a comparable 0.70 macro $F_1$ score, and the random order model performed worse with a 0.67 macro $F_1$ score. We provide our best DeBERTaV3 finetuned model publicly in a Huggingface repository\footnote{https://huggingface.co/ctu-aic/deberta-v3-large-AVeriTeC-nli}.

\subsubsubsection*{LLM Classifiers}
\todo{describe - likert motivation, GPT4o, Claude 3?}
\todo{try "opensource" LLMs? - maybe own section in an Appendix?}
To utilize the chain-of-thoughts abilities of large language models (LLMs), such as GPT4o~\cite{openai2024gpt4o} or Claude 3.5 Sonnet~\cite{anthropic2024claude35sonnet}, we tried to use them also for the classification task. Because we wanted to be able to perform ensembling (see section~\ref{subsubsec:ensembling}), we needed to output class probabilities. However, for the LLMs the math and numbers poses still a challenge~\cite{ahn-etal-2024-large}. To compensate the weaker mathematical reasoning abilities, we use the Likert scale as a proxy for the class probabilities. The Likert scale was the standard one, and the model was asked (see system prompt in Appendix~\ref{appendix_sec:llms}) to provide the Likert scale rating for each class. During postprocessing, we then transformed those to number scores\footnote{strongly disagree: -2, disagree: -1, neutral: 0, agree: 1, strongly agree: 2} and then we used the softmax function to get the wanted class probabilities.

\todo{check}
We know that our approach uses closed-source models; however, we could not utilize the open-source models due to limited time. Because we want to support the usage of open-source models, we provide the evaluation on open-source (or at least open-weights) models such as Llama 3.1~\cite{meta2024llama31}, Command~R~\cite{cohere2024commandr} in the Appendix~\ref{appendix_sec:opensource_llms}.

\subsubsubsection*{Ensembling}
\label{subsubsec:ensembling}
\todo{describe ensembling - average, weighted average, stacking using logreg}

Motivated by the results of each model, which were good at classifying into one of the classes, we decided to try ensembling the models. The first ensembling method we tried was simple averaging of the class probabilities of the models.

The second method is a logical extension of the first one, tuning the weights used in average and allowing us to use different weights for different models. This method was motivated by LLM classifiers, which often output exact probabilities for multiple classes. In this case, we can use the second model as a tiebreaker with a smaller weight in the average. The weights were optimized using bounded \texttt{minimize\_scalar} function from the Scipy library~\cite{2020SciPy-NMeth}.

The last method we tried was stacking using logistic regression. However, this setup classified no labels from \textit{Not Enough Evidence} and \textit{Conflicting Evidence/Cherrypicking}, and we could not achieve reasonable results. For logistic regression, we used the scikit-learn library~\cite{scikit-learn}.

\subsubsubsection*{Conflicting Evidence/Cherrypicking Detection}
\todo{Binary deberta with custom loss function - weighted crossentropy loss}
\todo{Bryce's idea - tf-idf + randomforests?, future works-new paper (not exactly what we are doing now, but interesting)~\cite{jaradat2024contextawaredetectioncherrypickingnews}}

During the experiments, we discovered that classifying the \textit{Conflicting Evidence/Cherrypicking} class is the most challenging task. To overcome this problem, we tried to build a binary classifier that would output one if the input claim is cherrypicked and 0 otherwise. First, we tried to use the DeBERTaV3 Large model with basic cross-entropy loss (other experimental settings were the same as in section~\ref{subsubsec:concatenation}). However, due to a high imbalance in the dataset, the model was not able to learn the classification task. To tackle this issue, we came up with a custom weighted cross-entropy loss function that would penalize the model more for misclassifying the \textit{Conflicting Evidence/Cherrypicking} class\footnote{The used weights corresponds to the ratio of classes in the training dataset: [0.3, 5]}. This approach improved slightly the performance, but it was still not usable in our pipeline.

Another approach we tried due to the lack of large training sets for cherrypicking was using a simple tf-idf representation of just the input claim and then using a random forest classifier~\footnote{In both cases, we used implementation from the Scikit-learn library~\cite{scikit-learn}}. However, this approach still did not provide satisfactory results.

While writing this system description paper, we found an interesting research by~\citet{jaradat2024contextawaredetectioncherrypickingnews} that uses a radically different approach to detect cherrypicking in newspaper articles.


\subsubsection{Classification Evaluation}
\todo{table with comparative results + comments}

    \centering
    \setlength\tabcolsep{3pt} % default value: 6pt
    \resizebox{\columnwidth}{!}{%
    \begin{tabular}{lcccc}
        \toprule
        \textbf{Classifier} & \textbf{Acc} & \textbf{$F_1$} & \textbf{Prec.} & \textbf{{\footnotesize {Recall}}} \\
        \midrule
        GPT4o & \textbf{0.72} & 0.46 & 0.48 & 0.47 \\
        Claude 3.5 Sonnet & 0.64 & 0.49 & 0.50 & 0.52 \\
        DeBERTa & 0.63 & 0.39 & 0.40 & 0.41 \\
        DeBERTa - random@10 & 0.65 & 0.41 & 0.41 & 0.44 \\
        $0.5\cdot\mbox{DeBERTa}+0.5\cdot\mbox{GPT4o}$ & 0.70 & 0.43 & 0.41 & 0.45 \\
        $0.5\cdot\mbox{DeBERTa}+0.5\cdot\mbox{Claude}$ & 0.68 & 0.47 & 0.50 & 0.49 \\
        $0.3\cdot\mbox{DeBERTa}+0.7\cdot\mbox{GPT4o}$ & \textbf{0.72} & 0.45 & 0.45 & 0.46 \\
        $0.3\cdot\mbox{DeBERTa}+0.7\cdot\mbox{Claude}$ & 0.66 & \textbf{0.50} & \textbf{0.51} & \textbf{0.53} \\
        $0.1\cdot\mbox{DeBERTa}+0.9\cdot\mbox{GPT4o}$ & \textbf{0.72} & 0.39 & 0.46 & 0.43 \\
        $0.1\cdot\mbox{DeBERTa}+0.9\cdot\mbox{Claude}$ & 0.64 & 0.49 & 0.50 & 0.54 \\
        \midrule
        Llama 3.1 & 0.73 & 0.44 & 0.43 & 0.46 \\
        \bottomrule
    \end{tabular}
    }
    \caption{Evalution of the classifiers on the development set. $F_1$, Precision and Recall are computed as macro-averages. The random@10 suffix indicates that the classifier was tuned using 10 different random orders of QA pairs for each claim. GPT4o stands for the Likert classifier based on GPT-4o, Claude 3.5 Sonnet is the Likert classifier based on Claude 3.5 Sonnet, and DeBERTa is classifier based on DeBERTaV3 Large fine-tuned on \averitec{} gold evidence and labels.}
    \label{tab:nli}
