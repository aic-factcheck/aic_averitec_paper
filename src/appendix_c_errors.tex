%!TEX ROOT=../emnlp2023.tex

\section{Examples of errors}
\label{appendix_sec:errors}
\lstset{
    %framexleftmargin=5mm,
    basicstyle=\ttfamily\footnotesize\linespread{0.9}, % Smaller font with less spacing
    numbers=none, % Line numbers on the left
    stepnumber=1, % Line numbers for every line
    numbersep=5pt, % Space between line numbers and code
    tabsize=4, % Size of tabs
    showstringspaces=false, % Don't show spaces in strings
    breaklines=true, % Line wrapping
    breakatwhitespace=true,
    frame=lines, % Add a frame around the code
    captionpos=b, % Caption at the 
    breakindent=1em,
    frame=single
}

\begin{lstlisting}[caption={Example of a claim where our pipeline uses newspaper sources instead of official government sources.}, label={lst:gov_error}, breaklines=true, breakatwhitespace=false, columns=fullflexible]
Claim 479: Donald Trump said "When the anarchists started ripping down our statues and monuments, right outside, I signed an order immediately, 10 years in prison."

gold evidence example: 
    question: What was the law signed by Trump regarding damaging federal property?
    answer: Trump signed an executive order that authorizes a penalty of up to 10 years in prison for damaging federal property. It does not say that it will automatically be a 10 year prison sentence.,
    source_url: https://web.archive.org/web/20210224033536/https://www.whitehouse.gov/presidential-actions/executive-order-protecting-american-monuments-memorials-statues-combating-recent-criminal-violence/

pipeline evidence example: 
    question: Did Trump sign an order related to vandalism of statues and monuments?, 
    answer: Yes, Trump signed an executive order to prosecute those who damage national monuments, making it a punishable offense with up to 10 years in jail.,
    url: https://m.economictimes.com/news/international/world-news/trump-makes-vandalising-national-monuments-punishable-offence-with-up-to-10-yrs-jail/articleshow/76658610.cms
    
\end{lstlisting}

\begin{lstlisting}[caption={Example of a claim and questions showing that the last tends to be unrelated or redundant to fact-checking of the claim.}, label={lst:unrelated_questions}, breaklines=true, breakatwhitespace=false, columns=fullflexible]
Claim 295: Trump campaign asked Joe Biden to release a list of potential Supreme Court picks only after Ginsburg's passing
question 1: Did Joe Biden claim that the Trump campaign asked him to release a list of potential Supreme Court picks only after Ginsburg's passing?
question 2: Did the Trump campaign ask Joe Biden to release a list of potential Supreme Court picks before Ginsburg's passing?
question 3: When did Trump release his latest list of potential Supreme Court nominees?
question 4: Did Trump personally demand that Biden release a list of potential Supreme Court nominees before Ginsburg's death?
question 5: What did Trump say about Biden releasing a list of potential Supreme Court nominees during the Republican National Convention?
question 6: Did the Trump campaign issue a statement on September 17, 2020, regarding Biden releasing a list of potential Supreme Court nominees?
question 7: What did the Trump campaign's statement on September 9, 2020, say about Biden releasing a list of potential Supreme Court nominees?
question 8: Did Biden indicate in June 2020 that he might release a list of potential Supreme Court picks?
quetion 9: What reason did Biden give for not releasing a list of potential Supreme Court nominees?,
question 10: Did Biden pledge to nominate a Black woman to the Supreme Court?
    
\end{lstlisting}

\begin{lstlisting}[caption={Example of a claim where our pipeline did not exactly extract the answer.}, label={lst:extractive_error}, breaklines=true, breakatwhitespace=false, columns=fullflexible]
Claim #155 - Trump said 'there were fine people on both side' in far-right protests.
answer: "You had some very bad people in that group, but you also had people that were very fine people, on both sides.", 
answer_type: Extractive
url: https://www.theatlantic.com/politics/archive/2017/08/trump-defends-white-nationalist-protesters-some-very-fine-people-on-both-sides/537012/
scraped text: ... "You also had some very fine people on both sides," he said. The Unite the Right rally that sparked the violence in Charlottesville featured several leading names in the white-nationalist alt-right movement, and also attracted people displaying Nazi symbols. ...
\end{lstlisting}

\begin{lstlisting}[caption={An example of a claim where the evidence consists mainly of evidence from PolitiFact and Factcheck.org fact-checking articles leading to different predicted label than in the gold dataset}, label={lst:polifact_leakage}, breaklines=true, breakatwhitespace=false, columns=fullflexible]
Claim #483 - Donald Trump said "We have spent nearly $2.5 trillion on completely rebuilding our military, which was very badly depleted when I took office."
Gold Label: Not Enough Evidence
Predicted Label: Refuted
pipeline evidence example: 
    question: What is the total defense budget for the last four fiscal years under Trump?
    url: https://www.politifact.com/factchecks/2020/jan/10/donald-trump/trump-exaggerates-spending-us-military-rebuild/
    question: Did Trump spend $2.5 trillion specifically on rebuilding the military?
    url: https://www.factcheck.org/2020/07/trumps-false-military-equipment-claim/
    ...
\end{lstlisting}
\pagebreak
\begin{lstlisting}[caption={An example of a claims which differs in length.}, label={lst:different_lengths}, breaklines=true, breakatwhitespace=false, columns=fullflexible]
Claim #0 - In a letter to Steve Jobs, Sean Connery refused to appear in an apple commercial.
Gold Evidence:
    question: Where was the claim first published
    answer: It was first published on Sccopertino
    question: What kind of website is Scoopertino
    answer: Scoopertino is an imaginary news organization devoted to ferreting out the most relevant stories in the world of Apple, whether or not they actually occurred - says their about page

Claim #315 - The fastest Supreme Court justice ever confirmed in the U.S. was 47 days.
Gold Evidence:
    question: What is the quickest time a Supreme Court justice nomination has been confirmed in the United States?
    answer: John Paul Stevens waited the fewest number of days (19)-followed by the most recent nominee to the Court, Amy Coney Barrett (27).61
    question: What is the average number of days between a nomination for a Supreme Court justice and the final Senate vote?
    answer: Overall, the average number of days from nomination to final Senate vote is 68.2 days (or approximately 2.2 months), while the median is 69.0 days.62 Of the 9 Justices currently serving on the Court, the average number of days from nomination to final Senate vote is 72.1 days (or approximately 2.4 months), while the median is 73.0 days. Among the current Justices, Amy Coney Barrett waited the fewest number of days from nomination to confirmation (27), while Clarence Thomas waited the greatest number of days (99).
\end{lstlisting}