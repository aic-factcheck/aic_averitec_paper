%!TEX ROOT=../emnlp2023.tex

\section{Results and analysis}
\label{sec:results}
\todo{write}

\begin{table*}[h]
    \centering
    \begin{tabular}{l | c c c | c c c}
    \hline
    &\multicolumn{3}{c|}{\textbf{Dev Set Results}} & \multicolumn{3}{c}{\textbf{Test Set Results}}  \\
    \textbf{Pipeline Name} & \textbf{Q only} & \textbf{Q+A} & \textbf{AVeriTeC Score} & \textbf{Q only} & \textbf{Q+A} & \textbf{AVeriTeC Score} \\ \hline
    \textbf{mmr+gpt4o-dfewshot-atype}      & 0.46 & 0.29 & 0.42 & 0.46 & 0.32 & 0.50\\
    mmr+gpt4o-dfewshot-tiebrk-atype & 0.46 & 0.29 & 0.41 & 0.46 & 0.32 & 0.50\\
    \averitec{} baseline            & 0.24 & 0.19 & 0.09 & 0.24 & 0.20 & 0.11\\
    \hline
    submission\_dev\_gpt\_likert       & 0.45 & 0.27 & 0.39 \\
    dev\_subquery+gpt4o                & 0.45 & 0.28 & 0.40 \\
    
    dev\_mmr+gpt4o-dfewshot-mock       & 0.45 & 0.28 & 0.41 \\
    submission\_dev\_claude            & 0.43 & 0.28 & 0.35 \\
    submission\_dev\_claude\_likert   & 0.43 & 0.28 & 0.35 \\
    dev\_mmr+gpt4o-dfewshot-gpttiebreaking & 0.38 & 0.24 & 0.31 \\
    dev\_mmr+gpt4o-dfewshot            & 0.45 & 0.29 & 0.42 \\
    dev\_mmr+gpt4o-dfewshot-gpttie-10ev & 0.45 & 0.28 & 0.40 \\
    tuned weightavg gpt4o(?)+DeBERTa          & 0.45 & 0.27 & 0.39 \\
    tuned weightavg Claude(?)+DeBERTa          & 0.43 & 0.28 & 0.36 \\
    dev\_subquery+gpt4o-dfewshot       & 0.45 & 0.29 & 0.42 \\
    claude evidence - DeBERTa cls               & 0.43 & 0.28 & 0.33 \\
    gpt4o evidence - DeBERTa cls               & 0.45 & 0.28 & 0.36 \\
    dev\_mmr+gpt4o                     & 0.45 & 0.28 & 0.38 \\
    MMR+DynamicFewshot RAG            & & & & 0.46 & 0.31 & 0.49 \\
    AIC first test                     & & & & 0.45 & 0.30 & 0.47 \\
    mmr+claude-dynamic-fewshot          & & & & 0.42 & 0.30 & 0.46 \\
    
    \hline
    mmr+llama-dfewshot-tiebrk-atype & 0.46 & 0.27 & 0.36 & 0.47 & 0.29 & 0.42\\
    \bottomrule
    \end{tabular}
    \caption{Comparison of Pipeline Scores on Dev and Test Sets, AVeriTeC scores are @0.25}
    \label{tab:pipeline_scores}
\end{table*}
    

\subsection{API Costs}
During our experimentation July 2024, we have made around 9000 requests to OpenAI's \texttt{gpt-4o-2024-05-13} batch API, at a total cost of \$363.
This gives a mean cost estimate of \$0.04 per a single fact-check (or \$0.08 using the API without the batch discount) that can be further reduced using cheaper models, such as \texttt{gpt-4o-2024-08-06}.

We argue that such costs make our model suitable for further experiments alongside human fact-checkers whose time spent reading through each source and proposing each evidence by themselves would certainly come at a higher price.

Our successive experiments with LLaMa 3.1~\cite{meta2024llama31} show promising results as well, nearly achieving parity with GPT.
The use of open-source models such as LLaMa or Mistral allows running our pipeline on premise, without leaking data to a third party and billing anything else than the computational resources.
For further experiments, we are looking to integrate them into the attached Python library using VLLM~\cite{vllm}.

\subsection{Strenghts \& Error analysis}
In this section we provide results of an exploratory analysis of 20 randomly selected samples from the development set. We divide our analysis into three parts: pipeline strenghts and errors, and dataset errors which we came across during our reviewing of the samples.

\subsubsection*{Strenghts}
\todo{podle mě pěkné otázky, zdá se že dodržuje zadané zdroje a nebere informace ze sebe}
\subsubsection*{Pipeline Errors}
\todo{používání více neoficiálních zdrojů ("gov" chyba) než anotátoři - u nás více noviny s volnějším stylem vs gov s tabulkami apod, otázky navíc (udělali jsme 10), extractive answers error (ale ne vždy -> future works), ?často čerpá hodněkrát z jednoho zdroje?, nevyužití claim date, neschopnost analýzy tabulek? }
\subsubsection*{Dataset Errors}
\todo{politifact leakage (leads even to error when our pipeline gives support instead of NEE), občas divné otázky (obecné, outside knowledge etc.), různě dlouhé otázky/odpovědi u různých datapoint (nekonzistence + zbytečné informace v odpovědích -> snižují skóre)}

\todo{what with this?}
All of the classifiers and pipelines we have set up for our task have exposed a significant dip in $F_1$ label score for the \texttt{Not Enough Evidence} and \texttt{Conflicting Evidence/Cherrypicking} labels.


